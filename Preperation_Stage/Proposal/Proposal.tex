\documentclass{article} % For LaTeX2e
\usepackage{11785_project,times}
\usepackage{hyperref}
\usepackage{url}

\title{Relay IOS App Project Proposal: Why we need Relay}


\author{
Bernny Velasquez \\
\texttt{bernny.velasquez@vikings.berry.edu} \\
}

\begin{document}


\maketitle

\begin{abstract}
The Relay app is is intenended to be an IOS application where the user starts a timer that will notify their contacts if the timer is not deactivated in time. Its purpose is to connect and reduce alert time for users who engage in solo activities such as running, hiking, etc., and may become incapacitated. 

\end{abstract}

\section{Purpose and Objective}
The purpose of the Relay app is to reduce alert time and provide a safety connection for users who engage in solo activities such as running, hiking, etc., and may become incapacitated. The objective is to create thorough documentation, deploy the application on the iPhone App Store, and achieve user downloads with high satisifaction. 

\section{Scope} The scope of the project will include, but is not limited too, User Registration and Setup, Timer-Based Notification System, Emergency contact alert, Location Tracking, and Failsafe features. Its intention is to notify user-designated emergency contacts when the timer expires. It will utilize GPS to record and share the user's location at regular intervals or when an alert is triggered. What it won't cover is active search and rescue, Real-time monitering without user input, and medical dianostics or assistance. These things will be outside of the scope of the project and go beyond the intention of the application. 

\section{User Needs} The target users for this application include outdoor enthusiasts solo adventurers, and individuals participating in high-risk activities. These users often venture into remote or secluded areas where immediate help may not be accessible, making it crucial to have a reliable way to alert someone if they encounter an emergency. Hikers, runners, and others require additional safety emasures, as typically the people who should know as soon as possible do not particiapate in said activites. Lastyl it can provide that added protection for safety-conscious individuals looking for added protection during their routines. By addressing these needs, the project provides a dependable, user-friendly tool for enchancing personal safety and improving response times in emergencies. 
\newpage

\section{Preliminary Timeline}
Phase 1: Planning and Requirements Gathering (Weeks 1-2)
Milestones:
\begin{itemize}
\item Finalize project scope, objectives, and functionality.
\item Identify target users and their needs.
\item Develop a detailed project requirements document.
\item Create wireframes and mockups for the app interface.
\end{itemize}

Phase 2: Design and Prototyping (Weeks 3-5)
Milestones:
\begin{itemize}
\item Design the app’s user interface (UI) and user experience (UX).
\item Finalize the architecture for features such as timers, notifications, and location tracking.
\item Develop a clickable prototype to demonstrate basic functionality and flow.
\end{itemize}

Phase 3: Core Feature Development (Weeks 6-8)
Milestones:
\begin{itemize}
\item Implement user registration and profile setup features.
\item Develop the timer-based notification system.
\item Integrate GPS location tracking and sharing functionality.
\item Set up the notification delivery system (e.g., SMS, email, or app alerts).
\item Build failsafe features, such as “Are You OK?” prompts.
\end{itemize}

Phase 4: Backend Development and Integration (Weeks 9-11)
Milestones:
\begin{itemize}
\item Build and integrate the backend system for user data storage and notifications.
\item Implement secure authentication and two-factor verification.
\item Add offline functionality for timer and basic features.(MAYBE)
\end{itemize}

Phase 5: Testing and Quality Assurance (Weeks 12-14)
Milestones:
\begin{itemize}
\item Perform usability testing with a small group of target users.
\item Fix bugs and improve functionality based on feedback.
\item Test edge cases, such as poor internet connectivity and false alarm prevention.
\end{itemize}

Phase 6: Deployment and Launch (Weeks 15)
Milestones:
\begin{itemize}
\item Publish the app on app store.
\item Provide documentation and tutorials for new users.
\item Ensure a support system is in place for user feedback and issues.
\end{itemize}







% You should cite all sources mentioned in this proposal in the file 11785_project.bib
% If you don't wish to cite some of your sources inline (e.g. in the Related Work section) using \cite{}, just you nocite to add
% them to the references section at the end of your proposal like so.
%\nocite{Bengio+chapter2007}
%\nocite{Hinton06}

%%\bibliography{11785_project}
%\bibliographystyle{11785_project}

\end{document}
